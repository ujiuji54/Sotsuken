\documentclass{jarticle}
\usepackage{abstract}

\title{電子工学科卒業研究発表会の予稿の書き方} %タイトル
\author{電子 太郎}% 報告者
\adviser{高専 太郎}% 指導教官

%%平成29年度よりヘッダはwebでpost時に自動で付けるよう変更
% \kind{最終}%
%\kind{中間}%
%\year{28}% 年度
%\num{10}% No.
\thispagestyle{empty}
\pagestyle{empty}

\begin{document}%
\maketitle % タイトルページの出力

\section{はじめに}%

\section{RNN}

\section{実験方法}
実験環境構成図
BOSS DS-1図
モデル構成図
推論イメージ図

\section{結果}
DIST:5損失比較図
DIST:5出力波形比較図
DIST:スペクトル波形比較

\section{考察}

\begin{thebibliography}{99}%参考文献
\bibitem{ref:1}
Taro Denshi : ``How to write'', Jpn. J. KCCT,
 \textbf{6}, pp.100-200 (2001).
\bibitem{ref:2}
高専 太郎 : ``論文記述法'', 神戸出版, pp.51-200 (2001).
\end{thebibliography}

\end{document}
