\documentclass{jreport}		% 標準のクラスファイル
\usepackage{paper}			% 神戸高専卒論用のマクロ
\usepackage[dvipdfmx]{graphicx}		% eps等図を取り込むためのマクロ
%\usepackage{times}

\title{LSTMを用いた楽器エフェクタのシミュレータ構築}	% 卒業論文のタイトル
\author{氏本 大智}			% 報告者
\adviser{長谷 芳樹}			% 指導教官
\year{30}					% 年度

\abstract{
 現在,音響機器がディジタルでシミュレートされ,ソフトウェアとして実現される場面が多くなっている.しかし,シミュレートには専用のハードが必要になるケースが多く,ソフトウェア単体で動作するシステムは未だ実用化されていない.\\
 本研究では,PC のみで動作し,スウィープ音を入れるだけで音響フィルタを手軽にシミュレートできる,ディープラーニングを用いた音響フィルタのシミュレータ開発の検討を行った.\\
 本研究では、ギターの原音データとひずみエフェクタを通した音声データを用意し、ニューラルネットワークを用いたシミュレータを作成した。その後、作成したシミュレータから得られた出力と教師データ出力の波形の比較、FFT処理を加え、得られた周波数スペクトルの比較を行った。その結果、出力波形は似かよったものが得られたが、周波数スペクトルをみると、6000Hzから精度が落ちることがわかった。\\
 今後、人間の耳による聴取実験を行い、シミュレータの評価を確立したのちに、教師データの選定や、ハイパーパラメータの最適化などの検討が必要である。
}%

\begin{document}%
\pagenumbering{roman}
\maketitle		% タイトルページの出力
\tableofcontents	% 目次出力

\chapter{はじめに}
\pagenumbering{arabic}
現在,音響機器がディジタルでシミュレートされ,ソフトウェアとして実現される場面が多くなっている.しかし,シミュレートには専用のハードが必要になるケースが多く,ソフトウェア単体で動作するシステムは未だ実用化されていない.

本研究では,PC のみで動作し,スウィープ音を入れるだけで音響フィルタを手軽にシミュレートできる,ディープラーニングを用いた音響フィルタのシミュレータの構築を目的としている.

\chapter{楽器エフェクタのシミュレータ}

\chapter{音響におけるニューラルネットワーク}
今回の研究では、楽器エフェクタをニューラルネットワークの手法の1つを用いてシミュレートする。
ここでは、Recurrent Neural Networkの基礎的事項を述べ、その発展的な手法であり、今回使用するLong Short-Term Memoryについて説明する。

\section{再帰的ニューラルネットワーク(RNN: Recurrent Neural Network)}
音声データは(x(1),...x(t),...x(T))というT個のデータが1つの入力データ群となる時系列データである。
通常のニューラルネットワークでは,音声や文章,映像などの時系列データを扱うことはできないが,RNN を使うことで中間層が時間方向に展開され,時系列データにも対応できる.
今回の研究では,RNN におけるユニットの入出力ゲートを拡張し,長期依存を学習可能にしたLSTM(Long Short Term Memory) を使用した.

\begin{figure}[htbp]
 \begin{center}
  \includegraphics[width=100mm]{RNN.png}
 \end{center}
 \caption{RNN構造図}
 \label{fig:one}
\end{figure}

\section{長期依存ニューラルネットワーク(LSTM: Long Short Term Memory)}

\chapter{実験方法}
\section{教師データ作成}
教師データは,100 msの音声データ(waveファイル)を入力側と出力側をそれぞれ用意し,それら1対を1つのデータセットとして扱う.

\subsection{入力側音声データ}
入力側音声データには,ギターの原音を使用した.PCとオーディオインターフェースをUSBケーブル,オーディオインターフェースとギターをアナログオーディオケーブルで結線し,サンプリング周波数48000 Hz,量子分解能は16 bitに設定して録音を実施した.低音域から高音域まで無作為に演奏したデータを50秒分用意した.

\subsection{出力側音声データ}
出力側音声データには,用意した入力側音声データを,ギター用の歪みエフェクタの一種であるBOSS DS-1に通したものを使用する.
オーディオインターフェースを介し,DS-1への入力とDS-1からの出力音声の録音を同時に行った.録音環境の機材構成図を図に示す.

\begin{figure}[htbp]
 \begin{center}
  \includegraphics[width=120mm]{env.png}
 \end{center}
 \caption{録音環境}
 \label{fig:one}
\end{figure}

\begin{figure}[htbp]
 \begin{center}
  \includegraphics[width=20mm]{ds1.jpg}
 \end{center}
 \caption{BOSS DS-1}
 \label{fig:one}
\end{figure}

DS-1には、DIST、TONE、LEVELの3つのつまみを持つ。つまみの値を1~9の9段階で定義すると、本研究では、DISTのつまみをを1,5,9にした3パターンのデータを用意した。DIST以外のつまみは5で固定し、音量はオーディオインターフェースのプリアンプ部で調整を行った。

\section{学習}
ライブラリはTensorflowをバックエンドとしたKerasを使用し、Googleが提供するブラウザ開発環境 Google Colaboratoryにて学習を実行した。

\subsection{学習データの前処理}

\subsection{モデル構築}
時系列データである音声データを扱うため、LSTMを使用する。モデルの構成は、input units=64,hidden units=64,output units=1とした。

LSTMは、入力データ長と同じ長さの出力が得られるが、今回作成したモデルでは、入力側音声データを110 ms入力し、出力の末尾10 msを最終出力として使う。推論の際には、入力側音声データを10 msずつずらしながらモデルを適用することで完全な出力を得る。

損失関数は平均二乗誤差(Mean Square Error)を使用した。
教師データ出力側とモデルの出力の末尾10 msから損失を算出した。

\subsection{学習}
50 sのデータのうち40 sを学習用データ、10 sをテストデータとし、学習データの損失とテストデータの損失を算出しながら学習を行った。
16サンプルを1ステップ、100ステップを1epochとし、100epochの計算を行った.

\section{評価実験}
\subsection{出力波形比較}
学習に使用していないギター音声データを,DS-1,学習モデルに入力し,DS-1から出力された音声データの波形と,モデルから出力された波形を比較した.

\subsection{周波数スペクトル比較}
の実験で出力された波形にFFT処理を行い,DS-1から出力された音声データの波形と,モデルから出力された波形の周波数スペクトルを比較した.

\chapter{結果}
本章では、第章の実験によって得られた結果について述べる。

\section{学習結果}
用意したDIST1,5,9の3パターンの教師データ、モデルを使用し、100Epochsの計算を行った。Epochsと損失の関係を図、表に示す。
\begin{figure}[htbp]
 \begin{center}
  \includegraphics[width=150mm]{gain1_loss_hikaku.png}
 \end{center}
 \caption{gain目盛り1 教師データ、テストデータの損失比較}
 \label{fig:one}
\end{figure}

\begin{table}
  \begin{tabular}{c|cc} \hline
Step&Train Data Loss&Validation Data Loss \\ \hline
0&0.00160506&0.000352979 \\
1&0.000213528&0.000317913 \\
2&0.000190245&0.000240705 \\
3&0.00017848&0.000237906 \\
4&0.000132364&0.000221933 \\
5&0.000130621&0.000163573 \\
6&0.000105466&0.000184859 \\
7&0.000115878&0.000194017 \\
8&8.79E-05&0.00014488 \\
9&0.000189127&0.001602182 \\
10&0.000345975&0.000215513 \\
11&9.86E-05&0.000172736 \\
12&7.73E-05&0.000121131 \\
13&7.00E-05&0.000137317 \\
14&7.60E-05&0.000134342 \\
15&6.09E-05&0.000112027 \\
16&5.10E-05&0.000118797 \\
17&4.55E-05&8.47E-05 \\
18&3.93E-05&7.75E-05 \\
19&4.62E-05&8.47E-05 \\
20&5.24E-05&0.000100358 \\
21&6.22E-05&0.000106018 \\
22&3.65E-05&7.35E-05 \\
23&7.34E-05&0.000107002 \\
24&6.59E-05&9.72E-05 \\
25&4.51E-05&7.47E-05 \\
26&4.13E-05&9.64E-05 \\
27&5.12E-05&0.000111188 \\
28&4.85E-05&9.34E-05 \\
29&3.75E-05&7.17E-05 \\
30&4.58E-05&0.000101449 \\
31&0.000565476&0.000609122 \\
32&0.000220356&0.000384264 \\
33&0.000152444&0.000184091 \\
34&6.33E-05&0.000127893 \\
35&4.73E-05&9.80E-05 \\
36&4.53E-05&9.60E-05 \\
37&3.47E-05&9.57E-05 \\
38&4.84E-05&8.67E-05 \\
39&3.44E-05&8.52E-05 \\
40&3.83E-05&8.37E-05 \\
41&3.46E-05&8.01E-05 \\
42&3.60E-05&9.00E-05 \\
43&8.87E-05&0.000126452 \\
44&5.68E-05&8.09E-05 \\
45&3.63E-05&5.76E-05 \\
46&4.46E-05&6.79E-05 \\
47&3.50E-05&7.24E-05 \\
48&3.88E-05&0.000110139 \\
49&5.54E-05&0.000105566 \\
50&5.55E-05&9.12E-05 \\
51&3.89E-05&8.24E-05 \\
52&4.30E-05&0.000111584 \\
53&4.03E-05&8.25E-05 \\
54&4.77E-05&7.39E-05 \\
55&5.74E-05&0.00015019 \\
56&4.77E-05&7.57E-05 \\
57&3.47E-05&6.18E-05 \\
58&3.35E-05&7.24E-05 \\
59&3.93E-05&8.95E-05 \\
60&4.90E-05&5.49E-05 \\
61&4.54E-05&6.57E-05 \\
62&3.12E-05&6.31E-05 \\
63&3.67E-05&8.42E-05 \\
64&5.53E-05&0.000135653 \\
65&4.51E-05&7.68E-05 \\
66&4.21E-05&0.000130762 \\
67&7.15E-05&9.59E-05 \\
68&3.91E-05&7.35E-05 \\
69&4.18E-05&8.13E-05 \\
70&3.20E-05&6.24E-05 \\
71&4.15E-05&7.45E-05 \\
72&4.17E-05&8.26E-05 \\
73&4.44E-05&7.36E-05 \\
74&4.73E-05&6.51E-05 \\
75&4.50E-05&9.51E-05 \\
76&3.66E-05&8.66E-05 \\
77&3.54E-05&6.10E-05 \\
78&5.27E-05&6.53E-05 \\
79&3.38E-05&7.27E-05 \\
80&3.65E-05&8.05E-05 \\
81&5.75E-05&0.000100877 \\
82&3.55E-05&6.79E-05 \\
83&3.69E-05&6.51E-05 \\
84&3.76E-05&7.29E-05 \\
85&3.45E-05&7.94E-05 \\
86&4.25E-05&0.000129144 \\
87&8.18E-05&0.000195353 \\
88&4.74E-05&6.41E-05 \\
89&3.57E-05&5.64E-05 \\
90&2.74E-05&6.88E-05 \\
91&3.57E-05&9.00E-05 \\
92&3.67E-05&9.45E-05 \\
93&3.45E-05&0.000101003 \\
94&3.34E-05&6.86E-05 \\
95&3.27E-05&7.63E-05 \\
96&3.05E-05&9.04E-05 \\
97&4.46E-05&8.68E-05 \\
98&3.78E-05&7.02E-05 \\
99&4.96E-05&5.97E-05 \\ \hline
  \end{tabular}
\end{table}

\begin{figure}[htbp]
 \begin{center}
  \includegraphics[width=150mm]{gain5_loss_hikaku.png}
 \end{center}
 \caption{gain目盛り5 教師データ、テストデータの損失比較}
 \label{fig:one}
\end{figure}

\begin{table}
  \begin{tabular}{c|cc} \hline
Step&Train Data Loss&Validation Data Loss \\ \hline
0&0.004889861&0.001435508 \\
1&0.000744052&0.001292358 \\
2&0.000730644&0.002151775 \\
3&0.000759727&0.001037813 \\
4&0.000516023&0.001335694 \\
5&0.000581503&0.000953488 \\
6&0.000463676&0.001102789 \\
7&0.000416913&0.000824363 \\
8&0.000473962&0.000643782 \\
9&0.000414526&0.000745018 \\
10&0.000341153&0.000713648 \\
11&0.000407701&0.000859346 \\
12&0.000328151&0.000730773 \\
13&0.00035718&0.000763542 \\
14&0.000354273&0.000805871 \\
15&0.000297489&0.0006634 \\
16&0.000284389&0.000644895 \\
17&0.000243047&0.000671774 \\
18&0.00028088&0.00053688 \\
19&0.000244971&0.000550054 \\
20&0.000240547&0.000558295 \\
21&0.000272628&0.000651547 \\
22&0.000287118&0.0006701 \\
23&0.000298778&0.000716666 \\
24&0.000279819&0.000625924 \\
25&0.000265637&0.000500552 \\
26&0.00024686&0.000529027 \\
27&0.000242713&0.000484245 \\
28&0.000216448&0.000547365 \\
29&0.000308574&0.000525933 \\
30&0.000361822&0.000595148 \\
31&0.00028465&0.000527179 \\
32&0.000247558&0.000496814 \\
33&0.00024506&0.000540644 \\
34&0.000249534&0.00066011 \\
35&0.00023539&0.000647761 \\
36&0.000314367&0.000651432 \\
37&0.00025952&0.000538898 \\
38&0.000209661&0.000488401 \\
39&0.000212282&0.00056086 \\
40&0.000246562&0.000481986 \\
41&0.0002391&0.000519534 \\
42&0.000618688&0.000831275 \\
43&0.000284162&0.00072398 \\
44&0.00022603&0.000569386 \\
45&0.000258338&0.000739949 \\
46&0.000252274&0.000518608 \\
47&0.000221021&0.000541543 \\
48&0.000247741&0.000519179 \\
49&0.000251099&0.00042931 \\
50&0.000245777&0.000693919 \\
51&0.000331987&0.000542177 \\
52&0.000217924&0.000543535 \\
53&0.000247419&0.000586915 \\
54&0.000220274&0.000662473 \\
55&0.000225418&0.000542753 \\
56&0.000221106&0.000507839 \\
57&0.000223819&0.000511765 \\
58&0.000206551&0.000555992 \\
59&0.000234044&0.000638897 \\
60&0.00023607&0.00047275 \\
61&0.000214041&0.000518529 \\
62&0.000207452&0.000740037 \\
63&0.000222599&0.000669925 \\
64&0.000194841&0.000554937 \\
65&0.000217739&0.000691593 \\
66&0.000244407&0.000463404 \\
67&0.000231235&0.000455803 \\
68&0.000217875&0.000562175 \\
69&0.000207241&0.000479158 \\
70&0.000202159&0.000496308 \\
71&0.000305013&0.000500735 \\
72&0.00023932&0.000470841 \\
73&0.000239582&0.000525681 \\
74&0.000219321&0.000503637 \\
75&0.000208494&0.000587894 \\
76&0.000221513&0.000482938 \\
77&0.000192633&0.000451671 \\
78&0.000201512&0.000505388 \\
79&0.000198546&0.000450816 \\
80&0.000214268&0.000546847 \\
81&0.000220396&0.000585774 \\
82&0.000209849&0.000462698 \\
83&0.000226151&0.000523901 \\
84&0.000209637&0.000486445 \\
85&0.000233725&0.000454348 \\
86&0.000207779&0.000534305 \\
87&0.000249129&0.000543835 \\
88&0.000215401&0.000595839 \\
89&0.000162287&0.000497318 \\
90&0.000195022&0.000505558 \\
91&0.000222369&0.000650684 \\
92&0.000209206&0.000469517 \\
93&0.000213904&0.000603006 \\
94&0.000196948&0.000506937 \\
95&0.000225688&0.000613343 \\
96&0.000252643&0.00057468 \\
97&0.000221213&0.000527929 \\
98&0.000264768&0.000476609 \\
99&0.000223447&0.000545185 \\ \hline
  \end{tabular}
\end{table}

\begin{figure}[htbp]
 \begin{center}
  \includegraphics[width=150mm]{gain10_loss_hikaku.png}
 \end{center}
 \caption{gain目盛り9 教師データ、テストデータの損失比較}
 \label{fig:one}
\end{figure}

\begin{table}
  \begin{tabular}{c|cc} \hline
Step&Train Data Loss&Validation Data Loss \\ \hline
0&0.004371066&0.003043697 \\
1&0.000759486&0.001153236 \\
2&0.000640299&0.00108158 \\
3&0.0006079&0.001058804 \\
4&0.000775743&0.001252678 \\
5&0.00044388&0.000998803 \\
6&0.000435707&0.000986739 \\
7&0.000447211&0.000885568 \\
8&0.000439035&0.000816686 \\
9&0.000412046&0.000727394 \\
10&0.000367349&0.000585502 \\
11&0.000302804&0.000925947 \\
12&0.000308004&0.000608787 \\
13&0.000269993&0.00072153 \\
14&0.000379837&0.000647763 \\
15&0.000365632&0.000589616 \\
16&0.000417786&0.000992585 \\
17&0.00026646&0.000582271 \\
18&0.000272529&0.000697525 \\
19&0.000205124&0.000593352 \\
20&0.000269611&0.000546894 \\
21&0.000295828&0.000555101 \\
22&0.000229603&0.000599482 \\
23&0.000296179&0.000657793 \\
24&0.000301725&0.000583811 \\
25&0.000210636&0.000544931 \\
26&0.000226497&0.000496407 \\
27&0.000295127&0.000600613 \\
28&0.000239622&0.000580198 \\
29&0.000260026&0.000522202 \\
30&0.000240948&0.000469696 \\
31&0.000253004&0.000788571 \\
32&0.000437932&0.000807393 \\
33&0.000247104&0.000474404 \\
34&0.000206442&0.000578834 \\
35&0.000250729&0.000670884 \\
36&0.000244497&0.000643456 \\
37&0.000304898&0.000556043 \\
38&0.000217728&0.000512277 \\
39&0.000216928&0.000511966 \\
40&0.000221378&0.0005882 \\
41&0.000285384&0.000525731 \\
42&0.000224457&0.000533225 \\
43&0.000216155&0.00055278 \\
44&0.000214125&0.000527411 \\
45&0.000222938&0.000513057 \\
46&0.000282341&0.000570891 \\
47&0.0002583&0.000504511 \\
48&0.000195442&0.000544505 \\
49&0.000219991&0.00049247 \\
50&0.000247146&0.000618363 \\
51&0.000210352&0.000520906 \\
52&0.00023072&0.000597587 \\
53&0.000236277&0.000501816 \\
54&0.000245679&0.000408237 \\
55&0.000239251&0.000521492 \\
56&0.00020355&0.000533951 \\
57&0.000166262&0.000531644 \\
58&0.000191958&0.000658119 \\
59&0.000185058&0.000818428 \\
60&0.000194305&0.000568074 \\
61&0.000208612&0.00056979 \\
62&0.000241436&0.00053354 \\
63&0.000207308&0.000492455 \\
64&0.000221103&0.000514421 \\
65&0.000211333&0.00042526 \\
66&0.000192082&0.000641374 \\
67&0.000251961&0.000750981 \\
68&0.000301935&0.000558112 \\
69&0.000248095&0.000522636 \\
70&0.000204014&0.000465629 \\
71&0.000229827&0.000533853 \\
72&0.000204647&0.000580661 \\
73&0.000208323&0.000624949 \\
74&0.000220868&0.000504471 \\
75&0.000219498&0.000562008 \\
76&0.000232896&0.000549256 \\
77&0.00024018&0.000604114 \\
78&0.000220515&0.00068188 \\
79&0.000252392&0.00068151 \\
80&0.00023004&0.000822419 \\
81&0.000231819&0.000657189 \\
82&0.000228214&0.000540706 \\
83&0.000248164&0.000456805 \\
84&0.000193526&0.000418304 \\
85&0.000210376&0.000550775 \\
86&0.000213836&0.000502167 \\
87&0.000188316&0.000444256 \\
88&0.000231664&0.000655027 \\
89&0.000218846&0.000518907 \\
90&0.000217386&0.000579928 \\
91&0.000197123&0.000554434 \\
92&0.000233412&0.000584695 \\
93&0.00018866&0.000463876 \\
94&0.00017825&0.000573682 \\
95&0.000184455&0.000524485 \\
96&0.000187724&0.000629065 \\
97&0.000247089&0.000531696 \\
98&0.000253038&0.00061189 \\
99&0.000224676&0.000452522 \\ \hline
  \end{tabular}
\end{table}

\clearpage
\section{出力波形}
に示す結果から、最もvalidationlossが低くなったEpochsのモデルを用いて、推論し、教師データとの比較を行った。推論の際、入力データは学習に使用していないものを使用した。
比較のため、0~0.02 sを拡大表示した。

\begin{figure}[htbp]
 \begin{center}
  \includegraphics[width=150mm]{gain1_output_hikaku.png}
 \end{center}
 \caption{gain目盛り1 波形拡大比較}
 \label{fig:one}
\end{figure}

\begin{figure}[htbp]
 \begin{center}
  \includegraphics[width=150mm]{gain5_output_hikaku.png}
 \end{center}
 \caption{gain目盛り5 波形拡大比較}
 \label{fig:one}
\end{figure}

\begin{figure}[htbp]
 \begin{center}
  \includegraphics[width=150mm]{gain10_output_hikaku.png}
 \end{center}
 \caption{gain目盛り9 波形拡大比較}
 \label{fig:one}
\end{figure}

\clearpage
\section{スペクトル分析}
推論で得た出力と教師データにFFT処理を行い、比較した。図に示す。

\begin{figure}[htbp]
 \begin{center}
  \includegraphics[width=150mm]{gain1_fft_hikaku.png}
 \end{center}
 \caption{gain目盛り1 教師データ、回帰モデル出力波形周波数スペクトル比較}
 \label{fig:one}
\end{figure}

\begin{figure}[htbp]
 \begin{center}
  \includegraphics[width=150mm]{gain5_fft_hikaku.png}
 \end{center}
 \caption{gain目盛り5 教師データ、回帰モデル出力波形周波数スペクトル比較}
 \label{fig:one}
\end{figure}

\begin{figure}[htbp]
 \begin{center}
  \includegraphics[width=150mm]{gain10_fft_hikaku.png}
 \end{center}
 \caption{gain目盛り9 教師データ、回帰モデル出力波形周波数スペクトル比較}
 \label{fig:one}
\end{figure}

\clearpage
\chapter{考察}

\end{document}
