\documentclass{jreport}		% 標準のクラスファイル
\usepackage{paper}			% 神戸高専卒論用のマクロ
\usepackage[dvipdfmx]{graphicx}		% eps等図を取り込むためのマクロ
%\usepackage{times}

\title{LSTMを用いた楽器エフェクタのシミュレータ構築}	% 卒業論文のタイトル
\author{氏本 大智}			% 報告者
\adviser{長谷 芳樹}			% 指導教官
\year{30}					% 年度

\abstract{
 論文には,表示に続いて論文要旨を記載する。論文要旨は,結果を含めて論文の内容が理解できるように簡潔にまとめられなければならない。論文の構成は,電子工学実験実習で購入した「知的な科学・技術文章の書き方」のP.57に記載されている,分量の少ない場合の論文の場合を参考にすること。また,論文の記載に関しては,本説明を熟読の上,体裁を整えることはもちろんのこと,論文内容も十分に吟味すること。\\
 完成した論文は,A4フラットファイルに下図に示すようにな表示および背表紙を付け,提出すること。
}%

\begin{document}%
\pagenumbering{roman}
\maketitle		% タイトルページの出力
\tableofcontents	% 目次出力

\chapter{はじめに}
\pagenumbering{arabic}

\chapter{楽器エフェクタのシミュレータ}

\chapter{音響におけるニューラルネットワーク}
今回の研究では、楽器エフェクタをニューラルネットワークの手法の1つを用いてシミュレートする。
ここでは、Recurrent Neural Networkの基礎的事項を述べ、その発展的な手法であり、今回使用するLong Short-Term Memoryについて説明する。

\section{再帰的ニューラルネットワーク(RNN: Recurrent Neural Network)}
音声データは(x(1),...x(t),...x(T))というT個のデータが1つの入力データ群となる時系列データである。
通常のニューラルネットワークでは,音声や文章,映像などの時系列データを扱うことはできないが,RNN を使うことで中間層が時間方向に展開され,時系列データにも対応できる.
今回の研究では,RNN におけるユニットの入出力ゲートを拡張し,長期依存を学習可能にしたLSTM(Long Short Term Memory) を使用した.

\begin{figure}[htbp]
 \begin{center}
  \includegraphics[width=100mm]{RNN.png}
 \end{center}
 \caption{RNN構造図}
 \label{fig:one}
\end{figure}

\section{長期依存ニューラルネットワーク(LSTM: Long Short Term Memory)}

\chapter{実験方法}
\section{教師データ作成}
本実験ではギター用の歪みエフェクタの一種である,BOSS DS-1を使用した.
\begin{figure}[htbp]
 \begin{center}
  \includegraphics[width=120mm]{env.png}
 \end{center}
 \caption{録音環境}
 \label{fig:one}
\end{figure}

\section{学習}

\section{評価実験}
\subsection{出力波形比較}
学習に使用していないギター音声データを,DS-1,学習モデルに入力し,DS-1から出力された音声データの波形と,モデルから出力された波形を比較した.

\subsection{周波数スペクトル比較}
の実験で出力された波形にフーリエ変換処理を行い,DS-1から出力された音声データの波形と,モデルから出力された波形の周波数スペクトルを比較した.

\chapter{結果}
本章では、第章の実験によって得られた結果について述べる。

\section{学習}
用意した教師データ、モデルを使用し、100Epochsの計算を行った。
\begin{figure}[htbp]
 \begin{center}
  \includegraphics[width=150mm]{gain1_loss_hikaku.png}
 \end{center}
 \caption{gain目盛り1 教師データ、テストデータの損失比較}
 \label{fig:one}
\end{figure}

\begin{figure}[htbp]
 \begin{center}
  \includegraphics[width=150mm]{gain5_loss_hikaku.png}
 \end{center}
 \caption{gain目盛り5 教師データ、テストデータの損失比較}
 \label{fig:one}
\end{figure}

\begin{figure}[htbp]
 \begin{center}
  \includegraphics[width=150mm]{gain10_loss_hikaku.png}
 \end{center}
 \caption{gain目盛り10 教師データ、テストデータの損失比較}
 \label{fig:one}
\end{figure}

\clearpage
\section{出力波形}
に示す結果から、最もvalidationlossが低くなったEpochsのモデルを使用し、推論を行った。。

\begin{figure}[htbp]
 \begin{center}
  \includegraphics[width=150mm]{gain1_output_hikaku.png}
 \end{center}
 \caption{gain目盛り1 波形拡大比較}
 \label{fig:one}
\end{figure}

\begin{figure}[htbp]
 \begin{center}
  \includegraphics[width=150mm]{gain5_output_hikaku.png}
 \end{center}
 \caption{gain目盛り5 波形拡大比較}
 \label{fig:one}
\end{figure}

\begin{figure}[htbp]
 \begin{center}
  \includegraphics[width=150mm]{gain10_output_hikaku.png}
 \end{center}
 \caption{gain目盛り10 波形拡大比較}
 \label{fig:one}
\end{figure}

\clearpage
\section{スペクトル分析}

\begin{figure}[htbp]
 \begin{center}
  \includegraphics[width=150mm]{gain1_fft_hikaku.png}
 \end{center}
 \caption{gain目盛り1 教師データ、回帰モデル出力波形周波数スペクトル比較}
 \label{fig:one}
\end{figure}

\begin{figure}[htbp]
 \begin{center}
  \includegraphics[width=150mm]{gain5_fft_hikaku.png}
 \end{center}
 \caption{gain目盛り5 教師データ、回帰モデル出力波形周波数スペクトル比較}
 \label{fig:one}
\end{figure}

\begin{figure}[htbp]
 \begin{center}
  \includegraphics[width=150mm]{gain10_fft_hikaku.png}
 \end{center}
 \caption{gain目盛り10 教師データ、回帰モデル出力波形周波数スペクトル比較}
 \label{fig:one}
\end{figure}

\clearpage
\section{卒業論文のフォーマット}%
\begin{description}
\item[用紙サイズ] A4縦置きで横書き
\item[枚数] 本文は20ページ以上,30ページ以下とする。資料となるデータ、補足などは本文外の付録とする。(付録は本文分量には含めない)。
\item[余白] 上下右は20mm、左は25mm
\item[書式] 通常文書の使用フォントは10〜11ポイント程度とする。また、
1ページは44文字×40行程度を標準とする。
\item[提出形式] 論文を印刷しA4フラットファイルに綴じて1部提出のこと。また,後日,PDF形式の電子データとして論文を提出してもらう。なお,ファイル名は,「r+学籍番号.PDF」とする。
\item[その他] 論文には,論文要旨(A4, 1ページ),目次を必ず添付すること。
\end{description}

\section{論文の構成}
一般的な卒業論文は,
\begin{description}
\item[表紙] 前述の指定したフォーマットで表紙を作成し,A4フラットファイルに貼り付ける。
\item[論文要旨] 論文内容がわかるように結果を含めて説明する。
\item[目次]
\item[本文] はじめに(序論),理論,実験方法,実験結果,まとめ(結論)などで構成される。
\item[付録] 実験データ,プログラムなど論文の関係のある資料を添付する。
\item[参考文献] 卒業論文を執筆する上で参考にした資料を記載する。
\item[謝辞] 論文を執筆するにあたり,お世話になった先生や友人にお礼を述べる。
\end{description}
で構成される。詳細については「知的な科学・技術文章の書き方」P.52の卒業論文の書き方を参照のこと。


\section{論文執筆における諸注意}
\subsection{変数および数式の記述方法}
文中には、単位や変数などで英語を用いる場合がある。このとき単位、数学記号にはローマン体、変数には斜体を用いるのが通例である。これは、数式中でも同様である。また,数式には通し番号を付けて整理すること。\par
例1) [A],[cm] \par
例2) $J$ [A], $E$ [eV] \par
例3)
\begin{equation}
f(x) = \log x {\rm [A]}
\end{equation}

\end{document}
