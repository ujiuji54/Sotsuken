\documentclass{jarticle}
\usepackage{abstract}

\title{電子工学科卒業研究発表会の予稿の書き方} %タイトル
\author{電子 太郎}% 報告者
\adviser{高専 太郎}% 指導教官

%%平成29年度よりヘッダはwebでpost時に自動で付けるよう変更
% \kind{最終}%
%\kind{中間}%
%\year{28}% 年度
%\num{10}% No.
\thispagestyle{empty}
\pagestyle{empty}

\begin{document}%
\maketitle % タイトルページの出力

\section{はじめに}%
ここでは、卒業論文の書き方について述べる。まず、論文の目次
を考え、論文の概要を決定する。次に、論文に掲載する図面を作
成する。続いて、理論・原理など不変的な事象から文章を書き始
め、実験結果・計算結果、考察、まとめの順序で記述を進める。
そして、最後にはじめにを書いて論文を完成させる。また、謝辞
、参考文献、付録なども必要に応じて記述する。論文に掲載する
図面や文面は、著作権等を考慮しなければならない。そのため、
参考文献の図をコピーしたり、まったく同じ文面を論文に掲載
してはならない。論文はオリジナルでなけらばならない。
\subsection{文中の英語について}%
文中には、単位や変数などで英語を用いる場合がある。このとき
単位、数学記号にはローマン体、変数には斜体を用いるのが通例である。
これは、数式中でも同様である。\par
例1) [A],[cm] \par
例2) $J$ [A], $E$ [eV] \par
例3)
\begin{equation}
f(x) = \log x {\rm [A]}  
\end{equation}

\subsection{図表の書き方}%
図の下側には、タイトルを付け、できる限り条件などの説明を加える。
文面を読まなくとも、図面だけ見て、読者に何について書かれた図面
なのかわかるようにする。また、図中の文字は小さすぎたり大きすぎ
たりしないように留意すること\cite{ref:1}。\par%
次に、表は上側にタイトルおよび説明を加える。表では、できる限り
縦罫は用いないようにする。また、表罫は太線とし、横罫もできるだ
け少なくする。\par%
例)
\begin{table}[h]
  \begin{center}
  \caption{表の例}%
    \begin{tabular}{ccc} \hline
         & 回答1 & 回答2 \\ \hline
    問1 & 1 &  3 \\
    問2 & 5 &  4 \\  
    問3 & 3 &  3 \\ \hline
    \end{tabular}
  \end{center}
\end{table}

\subsection{参考文献について}
参考文献は以下のように記述すること。
\\
論文の場合\\
著者 : ``タイトル'', 雑誌名, 巻, 号, ページ数 (年号)\\
本の場合\\
著者 : ``タイトル'', 出版社, ページ数 (年号).\\

\begin{thebibliography}{99}%参考文献
\bibitem{ref:1}
Taro Denshi : ``How to write'', Jpn. J. KCCT,
 \textbf{6}, pp.100-200 (2001). 
\bibitem{ref:2}
高専 太郎 : ``論文記述法'', 神戸出版, pp.51-200 (2001). 
\end{thebibliography}

\end{document}
